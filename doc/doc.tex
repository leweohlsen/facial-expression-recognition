\documentclass[a4paper,english]{report}
    
    \title{Projekt Medieninformatik\\Facial Expresson Recognition\\ Using A State Of The Art CNN}
    \date{WS17/18}
    \author{Lewe Ohlsen\\ Fachhochschule Wedel}
    
    \usepackage[utf8]{inputenc}  
    \usepackage[inline]{enumitem}
    \usepackage{tikz}
    \usepackage{pgfgantt}

    \begin{document}
        \maketitle
        \newpage
            \tableofcontents
        \newpage
        \section{Introduction}
            The projects goal is to build a device to process live webcam
            images, detect faces and classify the emotion of each face.
            In order to get started with deep learning techniques and state 
            of the art neural networks, I am participating in a Facial 
            Expression Recognition 
            Challenge\footnote{https://www.kaggle.com/c/challenges-in-representation-learning-facial-expression-recognition-challenge/data} 
            on Kaggle. 
        \section{Roadmap}
            \begin{ganttchart}{1}{16}
                \gantttitle{Calendar Week}{16} \\
                \gantttitlelist{46,...,52,1,2,3,4,5,6,7,8,9}{1} \\
                
                \ganttbar{MNIST Tutorial}{1}{2} \\
                \ganttlinkedbar{Data preparation}{3}{3} \\
                \ganttbar{implementing a simple linear model}{3}{4} \\                                
                \ganttlinkedbar{implementing a CNN}{4}{10} \\                
                
                %\ganttlinkedbar{Task 2}{3}{7} \ganttnewline
                \ganttmilestone{processing Webcam images}{8} \ganttnewline
                \ganttlinkedbar{Tuning parameters}{10}{12}
                
                %\ganttlink{elem1}{elem2}            
                
            \end{ganttchart}
        \chapter{About the Dataset}
        
        The Dataset consists of 48x48 pixel grayscale images of faces. 
        These images have been centered automatically to show a similar 
        region of the face in every sample. All the images come classified
        with the facial expression visible.
        % TODO: explain what a model is
        Initially the dataset is split into two subsets: A training set 
        that is used to fit the model and a test set to evaluate the model
        and see how well it performs.

        Figure of classes and distribution…
        
        \chapter{Challenges}
        \section{Recognizing Faces}
        \section{Classifying Facial Expressions}
        \chapter{Feedforward Neural Networks}
        A feedforward neural network can be thought of as a function f(x) = y
        where x is a fixed-size vector that represents the input data. In this
        case, the image is represented as a … dimensional vector, each component
        being the grayscale intensity of a pixel [0..1]. y is a 7-dimensional vector,
        each component representing the computed probability of the image belonging 
        to a certain class. The function f is actually a computational graph (DAG)…

        \section{Perceptrons}

        \section{Cost Function}

        \section{Optimizers}
        \section{Gradient Descend}
        \section{Backpropagation}        
        \chapter{Convolutional Neural Networks}
        \section{Convolutions}
        \section{Filters}
        \section{Layers and Dropouts}
        \chapter{Processing Live Images}
    \end{document}